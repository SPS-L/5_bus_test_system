\usepackage[english]{babel}
\usepackage{graphicx}
\usepackage{psfrag}
\usepackage{amstext,epsfig,amssymb,amsbsy,amsmath}
\usepackage{bm}
\usepackage{caption}
\usepackage{subcaption}
%\usepackage[latin1]{inputenc}
\usepackage{sectsty}
\usepackage[margin=1in,headheight=13.6pt]{geometry}
\allsectionsfont{\sffamily}
\usepackage{fancyhdr}
\pagestyle{fancy}
\fancyhf{}
%\fancyhead[RE,RO]{}
%\fancyhead[RE,LO]{Guides and tutorials}
\fancyhead[LE,LO]{CUT}
\fancyfoot[RE,RO]{Page \thepage}
\renewcommand{\headrulewidth}{1pt}
\renewcommand{\footrulewidth}{1pt}
\selectlanguage{english}
\usepackage[shortlabels]{enumitem}
\usepackage{titlesec}
\newcommand{\ua}{\underline{a\!}\,}
\usepackage{steinmetz}
\usepackage{mdframed}
\usepackage{pifont}
\usepackage{float}
\usepackage[linktoc=all]{hyperref}
\hypersetup{
	colorlinks,
	citecolor=black,
	filecolor=black,
	linkcolor=black,
	urlcolor=black
}
\usepackage{multicol}
\usepackage{booktabs}
\usepackage{rotating}

\usepackage[makeroom]{cancel}


\newenvironment{warning}
{\par\begin{mdframed}[linewidth=2pt,linecolor=red]%
		\begin{list}{}{\leftmargin=1cm
				\labelwidth=\leftmargin}\item[\Large\color{red}\ding{43}]}
		{\end{list}\end{mdframed}\par}
	
\newenvironment{hint}
{\par\begin{mdframed}[linewidth=2pt,linecolor=blue]%
		\begin{list}{}{\leftmargin=1cm
				\labelwidth=\leftmargin}\item[\Large\color{blue}\ding{45}]}
		{\end{list}\end{mdframed}\par}
	
%% Units
\usepackage[
per-mode=fraction,
detect-all,
detect-display-math = true,
detect-inline-family = text, %% NEEDS CURRENT VERSION OF SIUNITX
detect-inline-weight = text
]{siunitx}
% some specific power system units (which are not standard SI units..)
\DeclareSIUnit\pu{pu}
\DeclareSIUnit\voltampere{VA}
\DeclareSIUnit\var{var}
% define the versor notation ('1<5?'). Usage \versor[unit]{magnitude}{angle}
\newcommand{\versor}[3][]{#2\phase{#3^\circ}\,\si{#1}}

\usepackage{tikz}
\usetikzlibrary{shapes, arrows, arrows.meta}
\usetikzlibrary{decorations.pathreplacing}
\usetikzlibrary{positioning}
\usetikzlibrary{decorations.pathmorphing}
\usetikzlibrary{decorations.mypathmorphing}
\usetikzlibrary{decorations.markings}
\usetikzlibrary{patterns}
% DO NOT INPUT ANY CIRCUITS LIBRARIES HERE; AS THERE IS A CLASH BETWEEN
% circuitikz and circuits
\usetikzlibrary{circuits, circuits.ee.IEC}
\usepackage[europeanresistors,americaninductors,smartlabels]{circuitikz}
%\input{../circuitsPowerSystemSymbols.sty}
% \usepackage[european]{circuitikz}
\usepackage{pgfplots}
% Load the library
%\usetikzlibrary{external}
% Enable the library !!!>>> MUST be in the preamble <<<!!!!
%\tikzexternalize

\usepackage{ifthen}
\newboolean{student-version}  
\newcommand{\mysolution}[1] {%
	\ifthenelse{\boolean{student-version}}{}{%
		\subsection*{Solution}
		#1%
		\newpage
	}%
}

\newcommand{\myexercise}[2] {%
		\subsection{#1}
		#2%
}

\usepackage{xsim}
\xsimsetup{
	%  exercise/within=chapter,
	% exercise/template=theorem ,
	exercise/within = section ,
	exercise/the-counter = \thesection.\arabic{exercise} ,
	%  exercise/the-counter=\thechapter.\arabic{exercise},
	path = {temp}
}

\DeclareExerciseProperty{exhint}
% we'll use a description list for the hints:
\newcommand\printexhints{%
	\begin{description}
		\ForEachUsedExerciseByType{%
			\def\ExerciseType{##1}%
			\def\ExerciseID{##2}%
			\GetExercisePropertyT{exhint}
			{%
				\item[\XSIMmixedcase{\GetExerciseName}~##3]
				####1%
			}%
		}%
	\end{description}
}

\newcommand\exhint[1]{\SetExerciseProperty{exhint}{#1}}


\DeclareExerciseProperty{exanswer}
% we'll use a description list for the answers:
\newcommand\printexanswers{%
	\begin{description}
		\ForEachUsedExerciseByType{%
			\def\ExerciseType{##1}%
			\def\ExerciseID{##2}%
			\GetExercisePropertyT{exanswer}
			{%
				\item[\XSIMmixedcase{\GetExerciseName}~##3]
				####1%
			}%
		}%
	\end{description}
}

\newcommand\exanswer[1]{\SetExerciseProperty{exanswer}{#1}}


\def\begs{\begin{split}}
\def\ends{\end{split}}
\def\begequ{\begin{equation}}
\def\endequ{\end{equation}}
\def\lab{\label}
\def\begdes{\begin{description}}
\def\enddes{\end{description}}
\def\begenu{\begin{enumerate}}
\def\begite{\begin{itemize}}
\def\endite{\end{itemize}}
\def\endenu{\end{enumerate}}
\def\ctscale{0.7}
\def\ctlinewidth{0.5pt}
